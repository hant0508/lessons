\documentclass[PDF,10pt,usenames,dvipsnames]{beamer}
\usepackage[utf8]{inputenc} % прикручиваем русский язык
\usepackage[russian]{babel}

\usepackage[cm-default]{fontspec} % выбор шрифта (компилить xelatex'ом)
\setmainfont{Century Schoolbook L}
\setsansfont{Arial}
\setmonofont{Consolas}

\usepackage{hyperref} % поддержка ссылок
\usepackage{tabularx} % нормальные таблицы 

\usepackage{xpatch} % настройка отступов в списках
\xpatchcmd{\itemize}{\def\makelabel}{\setlength{\itemindent}{-9pt}\setlength{\itemsep}{0pt}\def\makelabel}{}{}
\setbeamertemplate{itemize items}[circle] % маркер в списках

\renewcommand{\baselinestretch}{0.9} % межстрочный интервал
\setbeamersize{text margin left=2mm,text margin right=2mm} % отступы
\setbeamertemplate{navigation symbols}{} % отключение клавиш навигации

\usepackage{xcolor} % цвета
\usepackage{tikz}
\setbeamercolor{title}{fg=black}
\setbeamercolor{frametitle}{fg=black}

\usepackage{relsize} % вопросительный знак
\newcommand{\bigqm}[1][1]{\text{\larger[#1]{\textbf{?}}}}

\everymath{\displaystyle} % нормальные дроби

\usepackage{ragged2e} % включаем поддержку переносов
\justifying % включаем переносы

\newcommand{\cpp}[1]{\textcolor{blue!40!black}{\tt #1}} % код
\newcommand{\prblm}[1]{{\rm\bigqm[1]} {#1 \\} \vspace{10pt}} % задача (inline)
\newcommand{\inp}{\\ \vspace{5pt} {\bf Входные данные} \\ \vspace{5pt}} % заголовок input для задач
\newcommand{\out}{\\ \vspace{5pt} {\bf Результат работы} \\ \vspace{5pt}} % заголовок output для задач 

% таблица с примерами входных данных и результатами работы
\newenvironment{ex}{\\ \vspace{5pt}{\bf Пример} \\
\vspace{5pt} \tabularx{\textwidth}{|X|X|}
\hline Входные данные & Результат работы \\ \hline}{\endtabularx}

\begin{document}

\begin{frame}[t] 
	\frametitle{Первая программа} 
	\begin{itemize}
		\item	\cpp{\#include<iostream>} и \cpp{using namespace std;} --- поддержка вывода текста 
		\item	\cpp{int main()} --- точка входа; с неё начинается {\bf любая} программа 
		\item	\cpp{cout << "текст";} --- выводит \cpp{текст} на экран 
		\item	\cpp{endl} --- перевод строки 
	\end{itemize}
\end{frame}

\begin{frame}[t]
	\frametitle{Задачи} 
	\prblm{Напишите программу, которая выводит на экран ваше имя.} 
	\prblm{Выведите на экран звёздочки в виде прямоугольного треугольника. \\ *\\ ** \\ *** \\ **** \\ *****} 
	\prblm{Вычислите, используя арифметические операции и скобки: \\ \vspace{5pt} $25+17$; \hspace{12pt} $\frac{5}{4}$; \hspace{12pt} $1+\frac{1}{1+\frac{1}{2}}$.} 
\end{frame}

\begin{frame}[t]
	\frametitle{Основные типы данных} 
	\begin{itemize}
		\item \cpp{int} --- целое число 
		\item \cpp{float} --- число с плавающей точкой 
		\item \cpp{string} --- строка 
		\item \cpp{char} --- символ 
		\item	\cpp{cin >> переменная;} --- считывает значение в \cpp{переменную}
	\end{itemize}
\end{frame}

\begin{frame}[t]
	\frametitle{Задачи} 
	\prblm{Сложите два целых числа.} 
	\prblm{Вычислите площадь квадрата по длине стороны.} 
	\prblm{Переведите заданное количество метров в километры.} 
	\prblm{Напечатайте последнюю цифру заданного натурального числа.} 
	\prblm{Вычислите $a^4$, использовав не более двух операций умножения.} 
	\prblm{Вычислите $a^{20}$, использовав не более пяти операций умножения.} 
\end{frame}

\begin{frame}[t]
	\frametitle{Материалы}
	{\bf \href{http://savthe.com/edu}{savthe.com/edu}}
	\begin{itemize}
		\item	VimC++ (запускать ярлык GVim) 
		\item	Учебник по С++ 
		\item	Шпаргалка по Vim 
	\end{itemize}
	{\bf \href{https://github.com/hant05080/lessons}{github.com/hant0508\textcolor{gray}0/lessons}}
\end{frame}

\begin{frame}[t]
	\frametitle{Бисер}
	В шкатулке хранится разноцветный бисер (или бусины). Все бусины имеют
	одинаковую форму, размер и вес. Бусины могут быть одного из N различных
	цветов. В шкатулке много бусин каждого цвета.  Требуется определить
	минимальное число бусин, которые можно не глядя вытащить из шкатулки так,
	чтобы среди них гарантированно были две бусины одного цвета. 
	\inp
	На вход подаётся одно натуральное число N --- количество цветов бусин ($1 \leq N \leq 10^9$). 
	\out
	Напечатайте одно целое число --- минимальное количество бусин.
	\begin{ex}
	3 & 4 \\ \hline
	\end{ex}
\end{frame}

\begin{frame}[t]
	\frametitle{Следующее и предыдущее}
	Напишите программу, которая считывает целое число и выводит текст с
	упоминанием следующего и предыдущего для него чисел. 
	\inp
	На вход подаётся целое число, не превосходящее $10^9$ по абсолютной величине.
	\out
	Напечатайте текст, аналогичный приведённому в примере.
	\begin{ex}
	42 & Следующее число после 42: 43 \newline Предыдущее число перед 42: 41 \\ \hline 
	\end{ex}	
\end{frame}

\begin{frame}[t]
	\frametitle{Магазин канцелярских товаров}
	Однажды, посетив магазин канцелярских товаров, Вася купил X карандашей, Y ручек
	и Z фломастеров. Известно, что цена ручки на 2 рубля больше цены карандаша и
	на 7 рублей меньше цены фломастера. Также известно, что стоимость карандаша
	составляет 3 рубля. Требуется определить общую стоимость покупки. 
	\inp
	На вход подаются 3 натуральных числа, каждое из которых не превосходит $10^9$
	\out
	Напечатайте одно целое число --- стоимость покупки в рублях.
	\begin{ex}
	1 1 1 & 20 \\ \hline
	1 2 3 & 49 \\ \hline
	\end{ex}
\end{frame}

\end{document}
