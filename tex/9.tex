\documentclass[PDF,10pt,usenames,dvipsnames,t,fragile]{beamer}
\usepackage[utf8]{inputenc} % прикручиваем русский язык
\usepackage[russian]{babel}

\usepackage[cm-default]{fontspec} % выбор шрифта (компилить xelatex'ом)
\setmainfont{Century Schoolbook L}
\setsansfont{Arial}
\setmonofont{Consolas}
\everymath{\displaystyle \tt} % математически выражения

\usepackage{hyperref} % поддержка ссылок
\usepackage{tabularx} % нормальные таблицы 

\usepackage{xpatch} % настройка отступов в списках
\xpatchcmd{\itemize}{\def\makelabel}{\setlength{\leftmargin}{3mm}\setlength{\itemsep}{0pt}\def\makelabel}{}{}
\setbeamertemplate{itemize items}[circle] % маркер в списках

\renewcommand{\baselinestretch}{0.9} % межстрочный интервал
\setbeamersize{text margin left=4mm,text margin right=2mm} % отступы
\addtolength{\headsep}{3mm}

\setbeamertemplate{navigation symbols}{} % отключение клавиш навигации

\usepackage{xcolor} % цвета
\usepackage{tikz}
\setbeamercolor{title}{fg=black}
\setbeamercolor{frametitle}{fg=black}

\usepackage{relsize} % вопросительный знак
\newcommand{\bigqm}[1][1]{\text{\rm\larger[#1]{\textbf{?}}}}

\usepackage{ragged2e} % включаем поддержку переносов
%\justifying % включаем переносы

\usepackage{minted} % настройка кода
\usemintedstyle{vs}
\renewcommand{\fcolorbox}[4][]{#4}
\newminted[code]{c_cpp.py:CppLexer -x}{tabsize=4, obeytabs, escapeinside=||}
%\newminted[code]{cpp}{tabsize=4, obeytabs, escapeinside=||}
\newmintinline[lcode]{c_cpp.py:CppLexer -x}{tabsize=4, obeytabs}

\newcommand{\prblm}[1]{{\bigqm[1]} {#1 \\} \vspace{-6pt} \\} % задача (inline)
\newcommand{\inp}{\vspace{4pt}\\ \vspace{4pt}{\bf Входные данные} \\} % заголовок input для задач
\newcommand{\out}{\vspace{4pt}\\ \vspace{4pt}{\bf Результат работы} \\} % заголовок output для задач 
\newcommand{\tb}{\\ \hline} % конец строки в таблице

% таблица с примерами входных данных и результатами работы
\setlength{\extrarowheight}{2pt}
\newenvironment{ex}{\vspace{4pt}\\ \vspace{4pt}{\bf Пример} \\
\tabularx{\textwidth}{|>{\tt}X|>{\tt}X|}
\hline \sf Входные данные & \sf Результат работы \tb}{\endtabularx}

\begin{document}

\begin{frame}[fragile]
	\frametitle{Первая программа}
	\begin{code}
#include <iostream> // подключаем библиотеку ввода/вывода текста
using namespace std; // it's a kind of magic :)

int main() // начало программы (т.н. точка входа)
{
	cout << "Hello, world!" << endl; // вывод текста
                                     // и перенос строки
}
	\end{code}
	Обратите внимание на синтаксис. Директива \lcode{#include} начинается с
	символа \lcode{#}, строковые константы (текст) записываются в кавычках, в
	конце большинства команд ставится точка с запятой.
\end{frame}

\begin{frame}
	\frametitle{Задачи}
	\prblm{Напишите программу, которая выводит на экран ваше имя.}
	\prblm{Выведите на экран звёздочки в виде прямоугольного треугольника. \\ *\\ ** \\ *** \\ **** \\ *****}
	\prblm{Вычислите, используя арифметические операции и скобки: \\ \vspace{5pt} $25+17$; \hspace{12pt} $\frac{5}{4}$; \hspace{12pt} $1+\frac{1}{1+\frac{1}{2}}$.}
\end{frame}

\begin{frame}[fragile]
	\frametitle{Основные типы данных}
	\begin{code}
#include <iostream>
using namespace std;
int main()
{
	int i = 42; // целое число
	double d = 3.1415; // вещественное число
	string s = "Hello!"; // строка
	char c = '+'; // символ (обратите внимание на кавычки)
	bool b = true; // логический тип (true / false)

	int a; // создали целочисленную переменную
	cin >> a; // и попросили ввести число c клавиатуры
}
	\end{code}
	Обратите внимание на направление "стрелочек": \\
	\lcode{cin >> a; // Ввод с клавиатуры в переменную а}
	\lcode{cout << a; // Вывод на экран значения переменной а}
\end{frame}

\begin{frame}
	\frametitle{Задачи}
	\prblm{Сложите два целых числа.}
	\prblm{Вычислите площадь квадрата по длине стороны.}
	\prblm{Переведите заданное количество метров в километры.}
	\prblm{Напечатайте последнюю цифру заданного натурального числа.}
	\prblm{Вычислите $a^4$, использовав не более двух операций умножения.}
	\prblm{Вычислите $a^{20}$, использовав не более пяти операций умножения.}
\end{frame}

\begin{frame}
	\frametitle{Бисер}
	В шкатулке хранится разноцветный бисер или бусины). Все бусины имеют
	одинаковую форму, размер и вес. Бусины могут быть одного из N различных
	цветов. В шкатулке много бусин каждого цвета.  Требуется определить
	минимальное число бусин, которые можно не глядя вытащить из шкатулки так,
	чтобы среди них гарантированно были две бусины одного цвета.
	\inp
	На вход подаётся одно натуральное число $N$ -- количество цветов бусин ($1 \leq N \leq 10^9$).
	\out
	Напечатайте одно целое число -- минимальное количество бусин.
	\begin{ex}
	3 & 4 \tb
	\end{ex}
\end{frame}

\begin{frame}
	\frametitle{Следующее и предыдущее}
	Напишите программу, которая считывает целое число и выводит текст с
	упоминанием следующего и предыдущего для него чисел.
	\inp
	На вход подаётся целое число, не превосходящее $10^9$ по абсолютной величине.
	\out
	Напечатайте текст, аналогичный приведённому в примере.
	\begin{ex}
	42 & Следующее число после 42: 43 \newline Предыдущее число перед 42: 41 \tb
	\end{ex}
\end{frame}

\begin{frame}
	\frametitle{Магазин канцелярских товаров}
	Однажды, посетив магазин канцелярских товаров, Вася купил $X$ карандашей, $Y$ ручек
	и $Z$ фломастеров. Известно, что цена ручки на 2 рубля больше цены карандаша и
	на 7 рублей меньше цены фломастера. Также известно, что стоимость карандаша
	составляет 3 рубля. Требуется определить общую стоимость покупки.
	\inp
	На вход подаются 3 натуральных числа, не превосходящих $10^9$
	\out
	Напечатайте одно натуральное число -- стоимость покупки в рублях.
	\begin{ex}
	1 1 1 & 20 \tb
	1 2 3 & 49 \tb
	\end{ex}
\end{frame}

\begin{frame}
	\frametitle{Сумма цифр}
	Найдите сумму цифр трёхзначного натурального числа.
	\inp
	На вход подаётся трёхзначное натуральное число.
	\out
	Напечатайте одно натуральное число -- сумму его цифр.
	\begin{ex}
		100 & 1 \tb
		123 & 6 \tb
	\end{ex}
\end{frame}

\begin{frame}[fragile]
	\frametitle{Инструкции ветвления}
	\begin{code}
int i;
cin >> i;
if (i > 3) // условие
{
	// выполнится, если условие верно
	cout << "Введённое число больше трёх";
}
else
{
	// если условие неверно
	cout << "Введённое число НЕ больше трёх";
}
	\end{code}
\end{frame}

\begin{frame}
	\frametitle{Операторы сравнения}
	\begin{itemize}
		\item \lcode{==} \ равно ({\bf не путать с $=$})
		\item \lcode{!=} \ не равно
		\item \lcode{<} \ \ \ меньше
		\item \lcode{>} \ \ \ больше
		\item \lcode{<=} \ меньше либо равно
		\item \lcode{>=} \ больше либо равно
		\item \lcode{&&} \ логическое И
		\item \lcode{||} \ логическое ИЛИ
	\end{itemize}
\end{frame}

\begin{frame}
	\frametitle{Задачи}
	\prblm{Поменять местами значения двух переменных.}
	\prblm{Решить предыдущую задачу без дополнительной переменной.}
	\prblm{Вычислить модуль введённого числа.}
	\prblm{Определить, является ли введённое число чётным.}
	\prblm{Найти максимальное из двух чисел.}
	\prblm{Проверить, могут ли 3 заданных числа быть сторонами треугольника.}
\end{frame}

\begin{frame}
	\frametitle{Калькулятор}
	Напишите калькулятор, выполняющий одно из 4 арифметических действий над двумя заданными вещественными числами.
	\inp
	На вход подаётся вещественное число $A$, символ $S$ и вещественное число $B$
	($|A|,|B| \leq 1000, S \in \{\text{'+', '-', '*', '/'}\}$).
	\out
	Напечатайте одно число -- результат вычисления, либо сообщение об ошибке.
	\begin{ex}
	2+3 & 5 \tb
	3.14*2.72 & 8.5408 \tb
	42/0 & Ошибка \tb
	\end{ex}
\end{frame}

\begin{frame}
	\frametitle{МКАД}
	Длина Московской кольцевой автомобильной дороги -- {\bf 109} километров. Байкер Вася
	стартует с первого километра МКАД и едет со скоростью $v$ километров в час. На
	какой отметке он остановится через $t$ часов?
	\inp
	На вход подаются два целых числа $t$ и $v$ ($0 \leq t,v \leq 40000$).
	\out
	Напечатайте единственное число от 1 до 109 -- километр МКАД, на котором остановится Вася.
	\begin{ex}
		60 2 & 12 \tb
		109 42 & 1 \tb
		0 146 & 1 \tb
	\end{ex}
\end{frame}

\begin{frame}
	\frametitle{Счастливый билет}
	Вы пользуетесь общественным транспортом? Вероятно, вы расплачивались за проезд
	и получали билет с номером. Счастливым билетом называют такой билет с
	шестизначным номером, где сумма первых трех цифр равна сумме последних трех.
	Т.е. билет с номером 385916 – счастливый, т.к. 3+8+5=9+1+6. Вам требуется
	написать программу, которая проверяет счастливость билета.
	\inp
	На вход подаётся одно целое число $N$ ($0 \leq N < 10^6$).
	\out
	Напечатайте \lcode{<<YES>>}, если билет с номером N счастливый и
	\lcode{<<NO>>} в противном случае.
	\begin{ex}
		385916 & YES \tb
		123456 & NO \tb
	\end{ex}
\end{frame}

\begin{frame}
	\frametitle{Торт}
	На свой день рождения Петя купил красивый и вкусный торт, который имел
	идеально круглую форму. Петя не знал, сколько гостей придет на его день
	рождения, поэтому вынужден был разработать алгоритм, согласно которому он
	сможет быстро разрезать торт на $N$ равных частей. Следует учесть, что разрезы
	торта можно производить как по радиусу, так и по диаметру.
	Помогите Пете решить эту задачу, определив наименьшее число разрезов торта по
	заданному числу гостей.
	\inp
	На вход подаётся натуральное число $N$ – число гостей, включая самого виновника
	торжества ($N \leq 2\cdot10^9$).
	\out
	Напечатайте одно целое число -- минимальное возможное число разрезов торта.
	\begin{ex}
		2 & 1 \tb
		3 & 3 \tb
	\end{ex}
\end{frame}

\begin{frame}
	\frametitle{Ладья}
	Напомним, что в шахматах используется клеточная доска размером $8х8$, где
	располагаются шахматные фигуры, которые могут перемещаться по определенным
	правилам. В частности, {\it ладья} может перемещаться на любое расстояние, как по
	вертикали, так и по горизонтали.

	Требуется определить: может ли ладья выполнить ход из клетки с координатами
	($X_1$,$Y_1$) в клетку с координатами ($X_2$,$Y_2$) на стандартной шахматной доске?
	\inp
	На вход подаются 4 числа: начальная координата {\it ладьи} $X_1$,$Y_1$ и конечная
	-- $X_2$,$Y_2$. Гарантируется, что начальная и конечная координаты не совпадают.
	\out
	Напечатайте <<YES>>, если ход допустим и <<NO>> в противном случае.
	\begin{ex}
		4 3 \newline 7 3 & YES \tb
		4 3 \newline 6 1 & NO \tb
	\end{ex}
\end{frame}

\begin{frame}[fragile]
	\frametitle{Короткая запись арифметических операций}

	\begin{itemize}
		\item Вместо \lcode{a = a @ b}, где \lcode{@} -- знак арифметической операции,
	можно писать \lcode{a @= b}. Например, \lcode{a += 2}.
		\item Вместо \lcode{a = a + 1} и \lcode{a = a - 1} можно писать \lcode{a++} (или
	\lcode{++a}) и \lcode{a--} (или \lcode{--a}) соответственно.
		\item В логическую переменную можно записывать результат сравнения.
	\end{itemize}
	Например, вместо
	\begin{code}
if(a > 5) b = true;
else b = false;
	\end{code}
	Можно писать \\
	\lcode{b = (a > 5);}
\end{frame}

\begin{frame}[fragile]
	\frametitle{Инструкции циклов}
	Циклы используются в программировании, когда некоторое действие
	нужно повторять многократно. В языке C++ существует несколько видов циклов,
	два наиболее часто используемых -- \lcode{while} и \lcode{for}. Команды,
	выполняемые в цикле, называются {\bf телом цикла}; один проход по телу цикла
	называется {\bf итерацией}.
	\begin{code}
// Аналогично if, но выполнится не один раз, а несколько:
// до тех пор, пока верно условие
while(/*условие*/)
{
	// тело цикла
}

/*инициализация; условие; действие*/
for(int i = 0; i < 10; ++i) // выполнить 10 итераций
{
	// тело цикла
}
	\end{code}
	Цикл \lcode{for} удобно использовать, когда известно заранее, сколько итераций
	нужно сделать. \newline

%	{\large Операторы прерывания цикла:}
%	\begin{itemize}
%		\item \lcode{break;} -- завершить выполнение цикла
%		\item \lcode{continue;} -- перейти к следующей итерации
%	\end{itemize}
\end{frame}

\begin{frame}
	\frametitle{Логические выражения}
	\begin{itemize}
		\item \lcode{bool} -- логический тип данных
		\item \lcode{!} -- оператор отрицания
		\item \lcode{0 == false, !0 == true // любое число, кроме нуля}
		\item \lcode{a = !!a; // 0, если a была 0; иначе 1}
		\item Операторы сравнения возвращают результат логического типа
	\end{itemize}
\end{frame}

\begin{frame}
	\frametitle{Задачи}
	\prblm{Напечатайте n звёздочек, не вводя дополнительную переменную.}
	\prblm{Напечатайте все целые числа (квадраты чисел; чётные числа) 0 до n.}
	\prblm{Напечатайте все делители числа n.}
	\prblm{Проверьте, является ли введенное натуральное число степенью тройки.}
	\prblm{Найдите максимальное из n натуральных чисел.}
	\prblm{Напечатайте квадраты целых чисел от 0 до n, не используя умножение и
	вложенные циклы.}
	\prblm{Проверьте, является ли натуральное число n простым.}
\end{frame}

\begin{frame}
	\frametitle{Демо ОГЭ 2019}
	% нет, это не мне было лень менять задачу, она реально осталась такая же
	Напишите программу, которая в последовательности натуральных чисел определяет
	минимальное число, оканчивающееся на 4. Программа получает на вход количество
	чисел в последовательности, а затем сами числа. В последовательности всегда
	имеется число, оканчивающееся на 4. Количество чисел не превышает 1000.
	Введённые числа не превышают 30 000. Программа должна вывести одно число --
	минимальное число, оканчивающееся на 4.
	\begin{ex}
	4 \newline 24 13 14 34 & 14 \tb
	\end{ex}
\end{frame}

\begin{frame}
	\frametitle{Монетки}
	На столе лежат n монеток. Некоторые из них лежат вверх решкой, а некоторые --
	гербом. Определите минимальное число монеток, которые нужно перевернуть,
	чтобы все монетки были повернуты вверх одной и той же стороной.
	\inp
	На вход подаётся количество монеток n, а следом n чисел: 0, если монетка лежит решкой вверх, или 1, если гербом вверх.
	\out
	Напечатайте минимальное число монеток, которые нужно перевернуть.
	\begin{ex}
		5 \newline 1 0 1 1 0 & 2 \tb
		4 \newline 0 0 1 0 & 1 \tb
	\end{ex}
\end{frame}

\begin{frame}
	\frametitle{Слон}
	Напомним, что в шахматах используется клеточная доска размером 8$\times$8, где
	располагаются шахматные фигуры, которые могут перемещаться по определенным
	правилам. В частности, {\it слон} может перемещаться на любое расстояние по диагонали.

	Требуется определить: может ли слон выполнить ход из клетки с координатами
	($X_1$,$Y_1$) в клетку с координатами ($X_2$,$Y_2$) на стандартной шахматной доске?
	\inp
	На вход подаются 4 числа: начальная координата {\it слона} $X_1$,$Y_1$ и конечная
	-- $X_2$,$Y_2$. Гарантируется, что начальная и конечная координаты не совпадают.
	\out
	Напечатайте {\tt YES}, если ход допустим и {\tt NO} в противном случае.
	\begin{ex}
		5 4 \newline 7 2 & YES \tb
		5 4 \newline 4 6 & NO \tb
	\end{ex}
\end{frame}

\begin{frame}
	\frametitle{Шоколадка}
	Требуется определить, можно ли от шоколадки размером $n\times m$ долек
	отломить $k$ долек, если разрешается сделать один разлом по прямой между
	дольками (то есть разломить шоколадку на два прямоугольника).
	\inp
	На вход подаются 3 натуральных числа ($n,m \leq 40000, k \leq 2\cdot10^9$).
	\out
	Напечатайте {\tt YES}, если возможно отломить указанное число долек и
	{\tt NO} в противном случае.
	\begin{ex}
		3 2 4 & YES \tb
		3 2 1 & NO \tb
		2 2 6 & NO \tb
	\end{ex}
\end{frame}

\begin{frame}
	\frametitle{Тройки и пятёрки}
Определите, можно ли с использованием только операций <<прибавить 3>> и
	<<прибавить 5>> получить из числа 1 число $n$. Само число 1 получить можно,
	просто не применяя никаких операций.
	\inp
	На вход подаётся натуральное число $n \leq 2\cdot 10^9$.
	\out
	Напечатайте {\tt YES}, если указанным способом можно получить $n$ и {\tt NO} в
	противном случае.
	\begin{ex}
		1 & YES \tb
		3 & NO \tb
	\end{ex}
\end{frame}

% current

\begin{frame}
	\frametitle{Орешки}
Белочка собрала в лесу $n$ шишек c орешками. Белочка очень привередливо выбирала
	шишки, и брала только те, в которых ровно $m$ орешков. Также известно, что для
	пропитания зимой ей необходимо не менее $k$ орешков. Определите, хватит ли на
	зиму орешков белочке.
	\inp
	На вход подаются натуральное числа $n, m, k$.
	\out
	Напечатайте {\tt YES}, если белочке хватит орешков на зиму и {\tt NO} в
	противном случае.
	\begin{ex}
		4 5 20 & YES \tb
		4 5 21 & NO \tb
		3 2 1 & YES \tb
	\end{ex}
\end{frame}

\begin{frame}
	\frametitle{Внеземные гости}
 Недавно на поле фермера Джона были обнаружены следы приземления летающих
	тарелок.  Поле имеет форму круга радиусом $r_1$. По сообщениям журналистов были
	обнаружены два следа от летающих тарелок, имевшие форму кругов. Один из них
	имел радиус $r_2$, второй -- радиус $r_3$. Также сообщается, что они находились
	внутри поля и не пересекались, ни один из них не лежал внутри другого. При
	этом, они, возможно, касались друг друга и/или границы поля. Поскольку
	журналисты часто склонны преувеличивать масштабы событий, необходимо написать
	программу, которая будет проверять, могли ли иметь место события, описанные в
	газете. 
	\inp
	На вход подаются 3 натуральных числа $r_1, r_2, r_3$.
	\out
	Напечатайте {\tt YES}, если такая ситуация возможна и {\tt NO} в
	противном случае.
	\begin{ex}
		10 10 10 & NO \tb
		10 3 4 & YES \tb
	\end{ex}
\end{frame}

\begin{frame}
	\frametitle{От перестановки что-то меняется}
	Всем известно, что <<от перестановки слагаемых сумма не меняется>>. Однако,
	случается и так, что перестановка двух чисел приводит к более интересным
	последствиям. Пусть, например, заданы три числа: $a_1, a_2, a_3$. Рассмотрим
	равенство $a_1 + a_2 = a_3$. Оно может быть неверным (например, если $a_1 = 1,
	a_2 = 4, a_3 = 3$), однако может стать верным, если поменять некоторые числа
	местами (например, если поменять местами $a_2$ и $a_3$, оно обратится в
	равенство $1 + 3 = 4$). Ваша задача -- по заданным трем числам определить:
	можно ли их переставить так, чтобы сумма первых двух равнялась третьему. 
	\inp
	На вход подаются 3 натуральных числа $a_1, a_2, a_3$.
	\out
	Напечатайте {\tt YES}, если числа можно переставить так, что сумма первых двух
	равна третьему, и {\tt NO} в противном случае.
	\begin{ex}
		3 5 2 & YES \tb
		2 2 5 & NO \tb
	\end{ex}
\end{frame}

\begin{frame}
	\frametitle{Перепись}
	В доме живет $n$ человек. Однажды решили провести перепись всех жильцов данного
	дома и составили список, в котором указали возраст и пол каждого жильца.
	Требуется найти номер самого старшего жителя мужского пола.
	\inp
	На вход подаётся натуральное число $n$, а следом за ним -- $n$ строк с
	информацией о жильцах. Каждая строка содержит натуральное число $a$ ($1 \leq a
	\leq 100$) и символ $s$ ({\tt'M'} -- мужчина или {\tt'F'} -- женщина).
	\out
	Напечатайте порядковый номер самого старшего мужчины, либо -1, если жильцов
	мужского пола нет.
	\begin{ex}
		3 \newline 25 M \newline 100 F \newline 70 M & 3 \tb
	\end{ex}
\end{frame}

\begin{frame}
	\frametitle{Автобусная экскурсия}
	Для обзорной экскурсии по городу был заказан двухэтажный автобус высотой {\bf
	437} сантиметров. На экскурсионном маршруте встречаются N мостов.
	Организаторы обеспокоены тем, что автобус может не проехать под одним из них.
	Им удалось выяснить точную высоту каждого моста. Автобус может проехать под
	мостом, если высота моста превосходит высоту автобуса. Помогите организаторам
	узнать, закончится ли экскурсия благополучно, а если нет, то установить, где
	произойдет авария.
	\inp
	На вход подаётся число $n$, а следом за ним -- $n$ натуральных чисел (высоты мостов).
	\out
	Напечатайте {\tt No crash}, если экскурсия закончится благополучно, в противном
	случае -- {\tt Crash k}, где k -- номер моста, где произойдёт авария.
	\begin{ex}
		3 \newline 763 245 113 & Crash 2 \tb
	\end{ex}
\end{frame}

\begin{frame}[fragile]
	\frametitle{Массивы}
	\begin{itemize}
		\item Массив -- последовательность элементов одного типа с общим именем
		\item Обращение к элементу массива осуществляется через его индекс
		\item Объявление массива в общем виде: \lcode{type name[size];}
		\item Нумерация элементов массива начинается {\bf с нуля}
		\item Размером массива может быть только {\bf константа}
		\item Для объявления константы используется спецификатор \lcode{const}
	\end{itemize}

	Пример:

	\begin{code}
int a[10]; // массив из 10 целых чисел
a[3] = 42; // инициализация ~{\bf четвёртого}~ элемента массива
const int n = 8; // целочисленная константа
double b[n] // массив из 8 вещественных чисел
n = 12; // ~{\bf ошибка}~: изменение константы
a[10] = 5; // ~{\bf ошибка}~: выход за границу массива (UB)
int m = 17;
char c[m]; // ~{\bf ошибка}~: m — не константа (IDB)
	\end{code}
\end{frame}

\begin{frame}[fragile]
	% экранировать символ процента в коде не надо
	\frametitle{Случайные числа в C++}
	\begin{itemize}
		\item \lcode{rand();} -- случайное целое неотрицательное число
		\item \lcode{rand()%n;} -- случайное целое число от нуля до n-1
	\end{itemize}

		Пример: заполним массив случайными числами от -25 до 25

	\begin{code}
const int n = 10; // размер массива — константа
int a[n];
for (int i = 0; i < n; ++i) // i = 0,1,…,9
		a[i] = rand()%51 - 25; // rand()%51 от 0 до 50
// выведем содержимое массива на экран
for (int i = 0; i < n; ++i)
	cout << a[i] << ' '; // выводим через пробел
cout << endl; // после вывода массива переходим на новую строку
	\end{code}
\end{frame}

\begin{frame}
	\frametitle{Задачи}
	В начале решения каждой задачи объявите массив из 20 целых чисел, заполните
	его случайными числами от -50 до 50 и выведите на экран содержимое этого
	массива.

	\prblm{Вычислите сумму элементов массива.}
	\prblm{Вычислите сумму квадратов положительных элементов массива.}
	\prblm{Вычислите среднее арифметическое отрицательных элементов массива.}
	\prblm{Вычислите произведение элементов массива, значения которых попадают в отрезок [l, r] (где $0 \leq l < r < 20$).}
	\prblm{Замените в массиве все нечетные элементы значениями их индексов.}
	\prblm{Определите значение максимального элемента в массиве и его положение.}
	\prblm{В массиве переставьте 1-й и 2-й элементы, 3-й и 4-й, 5-й и 6-й и т.д.}
	\prblm{Замените нулями все элементы массива, являющиеся делителями максимального элемента.}
	\prblm{Переставьте минимальный и максимальный элементы массива.}
\end{frame}

\begin{frame}[fragile]
	\frametitle{Символы}
	\begin{code}
char a = 'A'; // объявление символьной переменной
// Символы хранятся в виде кодов, с ними можно работать, как с
char b = a + 3; // D                            целыми числами
char c = 42; // *
// В таблице ASCII цифры и буквы латинского алфавита расположены
if ('X' < 'Z') //                                     по порядку
	cout << (int)'X' << ' ' << (int)'Z' << ' ' << endl; // 88 90
cout << '9' - '4' << endl; // 5
// Вместо символа можно явно выводить его код и наоборот
cout << 'A' << ' ' << (int)'A' << endl; // A 65
cout << 61  << ' ' << (char)61 << endl; // 61 =
	\end{code}
\end{frame}

\begin{frame}
	\frametitle{Задачи}
	\prblm{Определите, является ли введённый символ цифрой.}
	\prblm{Определите, сколько из 10 введённых символов являются буквами.}
	\prblm{Если введённый символ является строчной буквой латинского алфавита, переведите его в верхний регистр.}
\end{frame}

\begin{frame}[fragile]
	\frametitle{Строки}
	\begin{code}
string s = "hello"; // объявление строки
int n = s.size(); // размер строки (n == 5)
// Работа со строкой аналогична работе с массивом из char
cout << s[3] << s[4] << s[2] << endl; // lol
s[0] = 'H';
// Выводить строку можно целиком
cout << s << endl; // Hello
// Строки можно складывать (конкатенация) и сравнивать
string w = ", world!"; s += w;
cout << s << ' ' << s.size() << endl; // Hello, world! 13
// Удаление из строки
s.erase(5, 2); // (с какого символа, сколько символов)
cout << s << ' ' << s.size() << endl; // Helloworld! 11
// Вставка в строку
s.insert(5, ", new "); // (c какого символа, строка для вставки)
cout << s << ' ' << s.size() << endl; // Hello, new world! 17
// cin считывает строку до первого пробела
getline(cin, s); // так можно считать строку с пробелами
	\end{code}
\end{frame}

\begin{frame}
	\frametitle{Задачи}
	\prblm{Замените в строке все символы \lcode{x} на \lcode{y}.}
	\prblm{Определите, каких букв в строке больше: маленьких или больших.}
	\prblm{Проверьте, верно ли расставлены скобки в строке. Например, в строках: \lcode{"g(5)-2", "2 + (a-(b+5))", "((2-3) + (c-d))"} скобки расставлены верно, а в строках \lcode{"(u-3(", "15+a)", "2 - ) 3 + (7) ("} -- нет.}
	\prblm{Замените в строке все символы \lcode{x} на число 58.}
\end{frame}

\begin{frame}[fragile]
	\frametitle{Динамические массивы}
Обычные массивы иногда оказываются неудобными. Часто размер данных неизвестен, а
	создание массивов <<с запасом>> безосновательно увеличивает требуемый объем
	оперативной памяти. Стандартный контейнер \lcode{vector} работает, как обычный
	массив, но предоставляет удобные функции для работы с данными и не требует
	предварительного задания количества элементов.
	\begin{code}
#include<vector>
// ...
vector<int> v; // объявление вектора: vector<TYPE> NAME;
cout << v.size() << endl; // 0 (вектор пуст)
v.push_back(42); // добавление элемента
v.push_back(23);
v.push_back(-34);
cout << v.size() << endl; // 3
v[1] += 7;
// проход по элементам вектора
for (int i = 0; i < v.size(); ++i)
	cout << v[i] << ' '; // 42 30 -34
	\end{code}
\end{frame}

\begin{frame}
	\frametitle{Контроперация}
Хакер Василий получил доступ к классному журналу и хочет заменить все свои
	минимальные оценки на максимальные. Напишите программу, которая заменяет
	оценки Василия, но наоборот: все максимальные -- на минимальные.
	\inp
	На вход подаются целые ненулевые числа -- оценки Василия. Признак окончания последовательности -- число 0.
	\out
	Напечатайте исправленные оценки, сохранив их порядок. Ноль выводить не надо.
	\begin{ex}
		1 3 3 3 4 0 & 1 3 3 3 1 \tb
		5 4 2 2 4 2 2 5 0 & 2 4 2 2 4 2 2 2 \tb
		17 23 -5 7 23 0 & 17 -5 -5 7 -5 \tb
	\end{ex}
\end{frame}

\begin{frame}
	\frametitle{Налоги}
 В государстве действуют несколько фирм. Прибыль i-ой фирмы равна $V_i$ рублей в год. У царя есть любимые фирмы, а есть нелюбимые, поэтому налог для всех фирм разный и назначается царем в индивидуальном порядке. Налог на i-ую фирму равен $p_i$ процентов. Собиратели статистики решили посчитать, с какой фирмы в казну идет наибольший доход. Помогите им в этой задаче.
	\inp
	На вход подаются целые неотрицательные числа $V_i$, признак окончания последовательности -- число -1. Далее идут числа $p_i$.
	\out
	Напечатайте одно число - номер фирмы, от которой государство получает наибольший налог. Если таких фирм несколько, выведите фирму с наименьшим номером.
	\begin{ex}
		1 2 -1 \newline 3 2 & 2 \tb
		100 1 50 -1 \newline 0 100 3 & 3\tb
	\end{ex}
\end{frame}

\begin{frame}
	\frametitle{ОГЭ-2}
	Ученики 4 класса вели дневники наблюдения за погодой и ежедневно записывали
	дневную температуру. Найдите самую низкую температуру за время наблюдения.
	\inp
	Программа получает на вход количество дней, в течение которых проводилось
	измерение температуры N $(1 \leq N \leq 31)$, затем для каждого дня вводится
	температура.
	\out
	Выведите минимальную температуру. Если температура опускалась ниже –15
	градусов, выведите YES, иначе выведите NO.
	\begin{ex}
	4 \newline -5 12 -2 8 & -5 \newline NO \tb
	\end{ex}
\end{frame}

\begin{frame}
	\frametitle{ОГЭ-3}
	Напишите программу, которая в последовательности натуральных чисел вычисляет
	сумму всех двузначных чисел, кратных 8.
	\inp
	Программа получает на вход натуральные числа, количество введённых чисел
	неизвестно, последовательность чисел заканчивается числом 0 (признак
	окончания ввода, не входит в последовательность). Количество чисел не
	превышает 1000. Введённые числа не превышают 30 000.
	\out
	Программа должна вывести одно число: сумму всех двузначных чисел, кратных 8.
	\begin{ex}
	17 16 32 160 0 & 48 \tb
	\end{ex}
\end{frame}

\begin{frame}
	\frametitle{Считалка}
	Для выбора водящего в детской игре N человек становятся в круг, после чего
	произносится считалка. На первом слове считалки указывается на первого
	человека в кругу, на втором слове -- на второго человека и т. д. После N-го
	человека снова идёт первый человек (все люди в кругу пронумерованы числами от
	1 до N, круг зацикливается, после человека с номером N идёт человек с номером
	1). Всего в считалке M слов. Определите, на какого человека придётся
	последнее слово считалки.
	\inp
	Программа получает на вход два целых положительных числа. Первое число N –
	количество людей в кругу. Второе число M -- количество слов в считалке. Оба
	числа не превосходят $10^9$.
	\out
	Программа должна вывести одно целое число от 1 до N -- номер человека в кругу
	на которого придётся последнее слово считалки.
	\begin{ex}
	10 25 & 5 \tb
	\end{ex}
\end{frame}

\begin{frame}[fragile]
	\frametitle{Функции}
	Решение любой задачи естественно стараться свести к решению нескольких
	маленьких подзадач. Если в программе приходится несколько раз выполнять один и
	тот же алгоритм, то имеет смысл выделить его в подпрограмму и вызывать ее при
	необходимости. Такая подпрограмма будет называться {\bf функцией}. \\

	Функция имеет следующий вид:
	\begin{code}
тип имя(список аргументов)
{
	тело функции
}
	\end{code}

	Аргументы перечисляются через запятую с указанием типов. Если аргументов нет,
	ставятся пустые скобки.
\end{frame}

\begin{frame}
	\frametitle{Функции: возвращаемое значение}
	Функции в языке C++ делятся по смыслу на 2 типа.
	\begin{itemize}
		\item Тип \lcode{void}: функция выполняет определённые действия (чтение
			данных, вывод на экран, и т.д.). Точка выхода из функции -- её последняя
			команда или \lcode{return;}
		\item Любой другой тип (\lcode{int, bool, float,} \dots): функция вычисляет
			некоторое значение и {\bf возвращает} его. Точка выхода из функции --
			\lcode{return <значение>;}
	\end{itemize}
	Функции типа \lcode{void} вызываются по её имени со списком аргументов:
	\lcode{	print("hello");}\\
	После имени функции круглые {\bf скобки ставятся обязательно}, даже если
	функция не имеет аргументов. Функция должна быть объявлена до того места, где
	она будет вызвана. \\

	При вызове функций других типов, как правило, сохраняется или используется
	как-то иначе их возвращаемое значение: \\
	\lcode{	int m = max(2, 3);}
\end{frame}

\begin{frame}[fragile]
	\frametitle{Функции: пример}
	\begin{code}
void print(string text)
{
	cout << text << endl;
	// return; — не обязателен
}
int max(int a, int b) // тип нужно указывать у каждого аргумента
{
	// если a > b, функция завершит работу и вернёт значение а
	if (a > b) return a;
	// здесь можно не писать else, т.к. мы попадём сюда только
	// в том случае, если условие не выполнилось
	return b;
}
int main()
{
	print("hello");
	cout << max(4, 2) << endl; // выводим возвращаемое значение
}
	\end{code}
\end{frame}

\begin{frame}
	\frametitle{Унарная система счисления}
В унарной системе счисления числа записываются как соответствующее
	количество "палочек". Например, число 3 записывается как {\tt |||}, а число 7 – как
	{\tt |||||||}. Напишите программу, которая переводит координаты точки в пространстве
	в унарную систему счисления. Перевод числа реализуйте в виде отдельной
	функции.
	\inp
	На вход программе подаются 3 целых числа $x, y, z \geq 0$.
	\out
	Напечатайте значения чисел в унарной системе, каждое с новой строки.
	\begin{ex}
		2 3 4 & x: || \newline y: ||| \newline z: |||| \tb
	\end{ex}
	{\bf *}Измените функцию перевода так, чтобы она работала и для отрицательных
	чисел.
\end{frame}

\begin{frame}
	\frametitle{Палиндромы}
Палиндромами называются такие числа, десятичная запись которых читается
	одинаково слева направо и справа налево. Выясните, какие из заданных чисел
	являются палиндромами. (Подсказка: реализуйте вспомогательную функцию
	разворота числа в десятичной записи.)
	\inp
	На вход программе подаётся натуральное число n, а следом за ним -- n чисел
	$x_i \geq 0$.
	\out
	Для каждого числа в отельной строке напечатайте информацию о том, является ли
	оно палиндромом.
	\begin{ex}
		3 \newline 9 146 123321 & 9: Yes \newline 146: No \newline 123321: Yes \tb
	\end{ex}
\end{frame}

\begin{frame}
	\frametitle{Суммы цифр}
	Учительница записала на доске целое число N. Вовочка подсчитал сумму цифр этого
	числа и записал ее ниже. С полученным числом он проделал то же самое, и
	продолжал выписывать числа до тех пор, пока два последних записанных числа не
	совпали. Ваша задача -- найти сумму S всех выписанных на доску чисел. Реализуйте
	подсчёт суммы цифр в числе в виде отдельной функции.
	\inp
	На вход подаётся натуральное число $N \leq 2\cdot 10^9$.
	\out
	Напечатайте натуральное число S.
	\begin{ex}
		34 & 48 \tb
		1234 & 1246 \tb
		987654 & 987711 \tb
	\end{ex}
\end{frame}

\end{document}
