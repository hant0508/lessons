\documentclass[PDF,10pt,usenames,dvipsnames,t,fragile]{beamer}
\usepackage[utf8]{inputenc} % прикручиваем русский язык
\usepackage[russian]{babel}

\usepackage[cm-default]{fontspec} % выбор шрифта (компилить xelatex'ом)
\setmainfont{Century Schoolbook L}
\setsansfont{Arial}
\setmonofont{Consolas}

\everymath{\displaystyle \tt} % математически выражения
\usepackage{nicefrac}

\usepackage{hyperref} % поддержка ссылок
\usepackage{tabularx} % нормальные таблицы 

\usepackage{xpatch} % настройка отступов в списках
\xpatchcmd{\itemize}{\def\makelabel}{\setlength{\leftmargin}{3mm}\setlength{\itemsep}{0pt}\def\makelabel}{}{}
\setbeamertemplate{itemize items}[circle] % маркер в списках

\renewcommand{\baselinestretch}{0.9} % межстрочный интервал
\setbeamersize{text margin left=4mm,text margin right=2mm} % отступы
\addtolength{\headsep}{3mm}

\usepackage{graphicx} % картинки
\usepackage{tikz}
\usepackage{xcolor} % цвета
\setbeamercolor{title}{fg=black}
\setbeamercolor{frametitle}{fg=black}

\usepackage{relsize} % вопросительный знак
\newcommand{\bigqm}[1][1]{\text{\rm\larger[#1]{\textbf{?}}}}

\usepackage{ragged2e} % включаем поддержку переносов
%\justifying % включаем переносы

\usepackage{minted} % настройка кода
\usemintedstyle{vs}
\renewcommand{\fcolorbox}[4][]{#4}
\newminted[code]{c_cpp.py:CppLexer -x}{tabsize=4, obeytabs, escapeinside=~~}
%\newminted[code]{cpp}{tabsize=4, obeytabs, escapeinside=~~}
\newmintinline[lcode]{c_cpp.py:CppLexer -x}{tabsize=4, obeytabs}

\newcommand{\prblm}[1]{{\bigqm[1]} {#1 \\} \vspace{-6pt} \\} % задача (inline)
\newcommand{\inp}{\vspace{4pt}\\ \vspace{4pt}{\bf Входные данные} \\} % заголовок input для задач
\newcommand{\out}{\vspace{4pt}\\ \vspace{4pt}{\bf Результат работы} \\} % заголовок output для задач 
\newcommand{\tb}{\\ \hline} % конец строки в таблице

% таблица с примерами входных данных и результатами работы
\setlength{\extrarowheight}{2pt}
\newenvironment{ex}{\vspace{4pt}\\ \vspace{4pt}{\bf Пример} \\
\tabularx{\textwidth}{|>{\tt}X|>{\tt}X|}
\hline \sf Входные данные & \sf Результат работы \tb}{\endtabularx}

% отключение клавиш навигации и нумерация слайдов
\setbeamertemplate{navigation symbols}{}
\addtobeamertemplate{navigation symbols}{}{\small\color{gray}\insertframenumber\hspace{5pt}}
