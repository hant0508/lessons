\documentclass[PDF,10pt,usenames,dvipsnames,t,fragile]{beamer}
\usepackage[utf8]{inputenc} % прикручиваем русский язык
\usepackage[russian]{babel}

\usepackage[cm-default]{fontspec} % выбор шрифта (компилить xelatex'ом)
\setmainfont{Century Schoolbook L}
\setsansfont{Arial}
\setmonofont{Consolas}
\everymath{\displaystyle \tt} % математически выражения

\usepackage{hyperref} % поддержка ссылок
\usepackage{tabularx} % нормальные таблицы 

\usepackage{xpatch} % настройка отступов в списках
\xpatchcmd{\itemize}{\def\makelabel}{\setlength{\leftmargin}{3mm}\setlength{\itemsep}{0pt}\def\makelabel}{}{}
\setbeamertemplate{itemize items}[circle] % маркер в списках

\renewcommand{\baselinestretch}{0.9} % межстрочный интервал
\setbeamersize{text margin left=4mm,text margin right=2mm} % отступы
\addtolength{\headsep}{3mm}

\setbeamertemplate{navigation symbols}{} % отключение клавиш навигации

\usepackage{xcolor} % цвета
\usepackage{tikz}
\setbeamercolor{title}{fg=black}
\setbeamercolor{frametitle}{fg=black}

\usepackage{relsize} % вопросительный знак
\newcommand{\bigqm}[1][1]{\text{\rm\larger[#1]{\textbf{?}}}}

\usepackage{ragged2e} % включаем поддержку переносов
%\justifying % включаем переносы

\usepackage{minted} % настройка кода
\usemintedstyle{vs}
\renewcommand{\fcolorbox}[4][]{#4}
\newminted[code]{c_cpp.py:CppLexer -x}{tabsize=4, obeytabs, escapeinside=||}
%\newminted[code]{cpp}{tabsize=4, obeytabs, escapeinside=||}
\newmintinline[lcode]{c_cpp.py:CppLexer -x}{tabsize=4, obeytabs}

\newcommand{\prblm}[1]{{\bigqm[1]} {#1 \\} \vspace{-6pt} \\} % задача (inline)
\newcommand{\inp}{\vspace{4pt}\\ \vspace{4pt}{\bf Входные данные} \\} % заголовок input для задач
\newcommand{\out}{\vspace{4pt}\\ \vspace{4pt}{\bf Результат работы} \\} % заголовок output для задач 
\newcommand{\tb}{\\ \hline} % конец строки в таблице

% таблица с примерами входных данных и результатами работы
\setlength{\extrarowheight}{2pt}
\newenvironment{ex}{\vspace{4pt}\\ \vspace{4pt}{\bf Пример} \\
\tabularx{\textwidth}{|>{\tt}X|>{\tt}X|}
\hline \sf Входные данные & \sf Результат работы \tb}{\endtabularx}

\begin{document}
\begin{frame}[fragile]
	\frametitle{Функции}
	Решение любой задачи естественно стараться свести к решению нескольких
	маленьких подзадач. Если в программе приходится несколько раз выполнять один и
	тот же алгоритм, то имеет смысл выделить его в подпрограмму и вызывать ее при
	необходимости. Такая подпрограмма будет называться {\bf функцией}. \\
	
	Функция имеет следующий вид:
	\begin{code}
тип имя(список аргументов)
{
	тело функции
}
	\end{code}

	Аргументы перечисляются через запятую с указанием типов. Если аргументов нет,
	ставятся пустые скобки.
\end{frame}

\begin{frame}
	\frametitle{Функции: возвращаемое значение}
	Функции в языке C++ делятся по смыслу на 2 типа.
	\begin{itemize}
		\item Тип \lcode{void}: функция выполняет определённые действия (чтение
			данных, вывод на экран, и т.д.). Точка выхода из функции -- её последняя
			команда или \lcode{return;}
		\item Любой другой тип (\lcode{int, bool, float,} \dots): функция вычисляет
			некоторое значение и {\bf возвращает} его. Точка выхода из функции --
			\lcode{return <значение>;}
	\end{itemize}
	Функции типа \lcode{void} вызываются по её имени со списком аргументов:
	\lcode{	print("hello");}\\
	После имени функции круглые {\bf скобки ставятся обязательно}, даже если
	функция не имеет аргументов. Функция должна быть объявлена до того места, где
	она будет вызвана. \\

	При вызове функций других типов, как правило, сохраняется или используется
	как-то иначе их возвращаемое значение: \\
	\lcode{	int m = max(2, 3);}
\end{frame}

\begin{frame}[fragile]
	\frametitle{Функции: пример}
	\begin{code}
void print(string text)
{
	cout << text << endl;
	// return; — не обязателен
}
int max(int a, int b) // тип нужно указывать у каждого аргумента
{
	// если a > b, функция завершит работу и вернёт значение а
	if (a > b) return a;
	// здесь можно не писать else, т.к. мы попадём сюда только
	// в том случае, если условие не выполнилось
	return b;
}
int main()
{
	print("hello");
	cout << max(4, 2) << endl; // выводим возвращаемое значение
}
	\end{code}
\end{frame}

\begin{frame}
	\frametitle{Количество цифр}
 Дано три символа. Требуется определить, сколько из них являются цифрами. При
	решении данной задачи реализуйте функцию, которая возвращает 1, если символ --
	цифра, и 0 -- иначе.
	\inp
	На вход подаются 3 символа без разделителей.
	\out
	Напечатайте одно натуральное число -- количество цифр.
	\begin{ex}
		123 & 3 \tb
		A5! & 1 \tb
	\end{ex}
\end{frame}

\begin{frame}
	\frametitle{Сумма простых чисел}
	Даны N целых чисел. Определите, какие из чисел являются простыми и
	вычислите их сумму. Также определите, будет ли их сумма простым числом. При
	решении данной задачи реализуйте функцию, которая проверяет одно целое
	число на простоту.
	\inp
	На вход подаётся натуральное число N, далее ещё N целых неотрицательных чисел. 
	\out
	На первой строке напечатайте сумму простых чисел, на второй -- слово <<Yes>>,
	если их сумма -- простое число; иначе слово <<No>>.
	\begin{ex}
		3 \newline 3 5 11 & 19 \newline Yes \tb
		5 \newline 4 2 0 1 9 & 2 \newline Yes \tb
		2 \newline 8 8 & 0 \newline No \tb
	\end{ex}
\end{frame}

\begin{frame}
	\frametitle{Суммы цифр}
	Учительница записала на доске целое число N. Вовочка подсчитал сумму цифр этого
	числа и записал ее ниже. С полученным числом он проделал то же самое, и
	продолжал выписывать числа до тех пор, пока два последних записанных числа не
	совпали. Ваша задача -- найти сумму S всех выписанных на доску чисел. Реализуйте
	подсчёт суммы цифр в числе в виде отдельной функции.
	\inp
	На вход подаётся натуральное число $N \leq 2\cdot 10^9$.
	\out
	Напечатайте натуральное число S.
	\begin{ex}
		34 & 48 \tb
		1234 & 1246 \tb
		987654 & 987711 \tb
	\end{ex}
\end{frame}

\begin{frame}
	\frametitle{Подсчет букв}
 Дано три символа. Требуется определить, сколько из них являются буквами
	латинского алфавита. При решении данной задачи реализуйте функцию, которая
	возвращает 1, если символ -- цифра, и 0 -- иначе.
	\inp
	На вход подаются 3 символа без разделителей.
	\out
	Напечатайте одно натуральное число -- количество букв.
	\begin{ex}
		i23 & 1 \tb
		A5n & 2 \tb
		282 & 0 \tb
	\end{ex}
\end{frame}

\begin{frame}
	\frametitle{В одном шаге от счастья}
 Вова купил билет в трамвае 13-го маршрута и сразу посчитал суммы первых трёх
	цифр и последних трёх цифр номера билета (номер у билета шестизначный).
	Оказалось, что суммы отличаются ровно на единицу. <<Я в одном шаге от
	счастья>>, -- подумал Вова, -- <<или предыдущий или следующий билет точно
	счастливый>>. Прав ли он? 
	\inp
	На вход подаётся количество билетов \lcode{N}, далее \lcode{N} шестизначных
	номеров билетов. 
	\out
	Для каждого номера напечатайте \lcode{"Yes"}, если Вова был прав; иначе
	\lcode{"No"}
	\begin{ex}
		3 \newline 715068 \newline 445219 \newline 012200 & Yes \newline No \newline Yes \tb
	\end{ex}
\end{frame}

\begin{frame}
	\frametitle{Секрет}
 Вам в руки попала секретная записка на английском языке. Текст записки может
	быть любым, главное - код, заложенный в тексте. Чтобы расшифровать записку
	нужно посчитать количество букв «b» и «g» в записке (на любом регистре).  Если
	букв «b» больше, чем букв «g», то все плохо. Если букв «b» меньше, чем букв
	«g», то все хорошо. Ну, а если буквы содержатся в записке в одинаковом
	количестве, то пока не ясно, как дела пойдут. Напишите программу для
	расшифровки таких секретных записок. 
	\inp
	На вход подаются строки, состоящие из английских букв, цифр, пробелов и знаков
	препинания. 
	\out
	Напечатайте все строки в неизменном виде, после на новой строке выведите 
слово, определяющее тайный смысл записки: \\
<<GOOD>> -- если все хорошо; \\
<<BAD>> -- если все плохо; \\
<<NEUTRAL>> -- если пока не ясно, как пойдут дела.
\end{frame}

\begin{frame}
	\frametitle{Секрет}
	\leavevmode
	\begin{ex}
		It is rainy and I have bought umbrella & It is rainy and I have bought
		umbrella \newline BAD \tb
		It is rainy \newline and I have bought tea & It is rainy \newline and I
		have bought tea \newline NEUTRAL \tb
		My aunt Ann is greedy! & My aunt Ann is greedy! \newline GOOD \tb
	\end{ex}
\end{frame}

\end{document}
