\documentclass[PDF,10pt,usenames,dvipsnames,t,fragile]{beamer}
\usepackage[utf8]{inputenc} % прикручиваем русский язык
\usepackage[russian]{babel}

\usepackage[cm-default]{fontspec} % выбор шрифта (компилить xelatex'ом)
\setmainfont{Century Schoolbook L}
\setsansfont{Arial}
\setmonofont{Consolas}

\usepackage{hyperref} % поддержка ссылок
\usepackage{tabularx} % нормальные таблицы 

\usepackage{xpatch} % настройка отступов в списках
\xpatchcmd{\itemize}{\def\makelabel}{\setlength{\itemindent}{-9pt}\setlength{\itemsep}{0pt}\def\makelabel}{}{}
\setbeamertemplate{itemize items}[circle] % маркер в списках

\renewcommand{\baselinestretch}{0.9} % межстрочный интервал
\setbeamersize{text margin left=4mm,text margin right=2mm} % отступы
\addtolength{\headsep}{3mm}

\setbeamertemplate{navigation symbols}{} % отключение клавиш навигации

\usepackage{xcolor} % цвета
\usepackage{tikz}
\setbeamercolor{title}{fg=black}
\setbeamercolor{frametitle}{fg=black}

\usepackage{relsize} % вопросительный знак
\newcommand{\bigqm}[1][1]{\text{\rm\larger[#1]{\textbf{?}}}}

\everymath{\displaystyle} % нормальные дроби

\usepackage{ragged2e} % включаем поддержку переносов
%\justifying % включаем переносы

\usepackage{minted} % настройка кода
\usemintedstyle{vs}
\renewcommand{\fcolorbox}[4][]{#4}
\newminted[code]{cpp}{mathescape, tabsize=4, obeytabs, linenos}
\newmintinline[lcode]{cpp}{mathescape, tabsize=4, obeytabs}

\newcommand{\prblm}[1]{{\bigqm[1]} {#1 \\} \vspace{-6pt} \\} % задача (inline)
\newcommand{\inp}{\\ \vspace{5pt} {\bf Входные данные} \\ \vspace{5pt}} % заголовок input для задач
\newcommand{\out}{\\ \vspace{5pt} {\bf Результат работы} \\ \vspace{5pt}} % заголовок output для задач 
\newcommand{\tb}{\\ \hline} % конец строки в таблице

% таблица с примерами входных данных и результатами работы
\newenvironment{ex}{\\ \vspace{5pt}{\bf Пример} \\
\tabularx{\textwidth}{|X|X|}
\hline Входные данные & Результат работы \tb}{\endtabularx}

\begin{document}

\begin{frame}[fragile]
	\frametitle{Первая программа}
	\begin{itemize}
		\item	\lcode{#include <iostream>} и \lcode{using namespace std;} --- для вывода текста
		\item	\lcode{int main()} --- точка входа; с неё начинается {\bf любая} программа
		\item	\lcode{cout << "текст";} --- выводит \lcode{текст} на экран
		\item	\lcode{endl;} --- перевод строки
	\end{itemize}
\end{frame}

\begin{frame}
	\frametitle{Задачи}
	\prblm{Напишите программу, которая выводит на экран ваше имя.}
	\prblm{Выведите на экран звёздочки в виде прямоугольного треугольника. \\ *\\ ** \\ *** \\ **** \\ *****}
	\prblm{Вычислите, используя арифметические операции и скобки: \\ \vspace{5pt} $25+17$; \hspace{12pt} $\frac{5}{4}$; \hspace{12pt} $1+\frac{1}{1+\frac{1}{2}}$.}
\end{frame}

\begin{frame}
	\frametitle{Основные типы данных} 
	\begin{itemize}
		\item \lcode{int} --- целое число
		\item \lcode{float} --- число с плавающей точкой
		\item \lcode{string} --- строка
		\item \lcode{char} --- символ
		\item	\lcode{cin >> переменная;} --- считывает значение в \lcode{переменную}
	\end{itemize}
\end{frame}

\begin{frame}
	\frametitle{Задачи}
	\prblm{Сложите два целых числа.}
	\prblm{Вычислите площадь квадрата по длине стороны.}
	\prblm{Переведите заданное количество метров в километры.}
	\prblm{Напечатайте последнюю цифру заданного натурального числа.}
	\prblm{Вычислите $a^4$, использовав не более двух операций умножения.}
	\prblm{Вычислите $a^{20}$, использовав не более пяти операций умножения.}
\end{frame}

\begin{frame}
	\frametitle{Материалы}
	{\bf \href{http://savthe.com/edu}{savthe.com/edu}}
	\begin{itemize}
		\item	VimC++ (запускать ярлык GVim) 
		\item	Учебник по С++ 
		\item	Шпаргалка по Vim 
	\end{itemize}
	{\bf \href{https://github.com/hant05080/lessons}{github.com/hant0508\textcolor{gray}0/lessons}}
\end{frame}

\begin{frame}
	\frametitle{Бисер}
	В шкатулке хранится разноцветный бисер или бусины). Все бусины имеют
	одинаковую форму, размер и вес. Бусины могут быть одного из N различных
	цветов. В шкатулке много бусин каждого цвета.  Требуется определить
	минимальное число бусин, которые можно не глядя вытащить из шкатулки так,
	чтобы среди них гарантированно были две бусины одного цвета. 
	\inp
	На вход подаётся одно натуральное число N --- количество цветов бусин ($1 \leq N \leq 10^9$). 
	\out
	Напечатайте одно целое число --- минимальное количество бусин.
	\begin{ex}
	3 & 4 \tb
	\end{ex}
\end{frame}

\begin{frame}
	\frametitle{Следующее и предыдущее}
	Напишите программу, которая считывает целое число и выводит текст с
	упоминанием следующего и предыдущего для него чисел. 
	\inp
	На вход подаётся целое число, не превосходящее $10^9$ по абсолютной величине.
	\out
	Напечатайте текст, аналогичный приведённому в примере.
	\begin{ex}
	42 & Следующее число после 42: 43 \newline Предыдущее число перед 42: 41 \tb 
	\end{ex}	
\end{frame}

\begin{frame}
	\frametitle{Магазин канцелярских товаров}
	Однажды, посетив магазин канцелярских товаров, Вася купил \lcode{x} карандашей, \lcode{y} ручек
	и \lcode{z} фломастеров. Известно, что цена ручки на 2 рубля больше цены карандаша и
	на 7 рублей меньше цены фломастера. Также известно, что стоимость карандаша
	составляет 3 рубля. Требуется определить общую стоимость покупки. 
	\inp
	На вход подаются 3 натуральных числа, не превосходящих $10^9$
	\out
	Напечатайте одно натуральное число --- стоимость покупки в рублях.
	\begin{ex}
	1 1 1 & 20 \tb
	1 2 3 & 49 \tb
	\end{ex}
\end{frame}

\begin{frame}
	\frametitle{Сумма цифр}
	Найдите сумму цифр трёхзначного натурального числа.
	\inp
	На вход подаётся трёхзначное натуральное число.
	\out
	Напечатайте одно натуральное число --- сумму его цифр.
	\begin{ex}
		100 & 1 \tb
		123 & 6 \tb
	\end{ex}
\end{frame}

\begin{frame}[fragile]
	\frametitle{Инструкции ветвления}
	\begin{code}
if (/* условие */)
{
	// некоторые действия
	// выполнятся, если условие верно
}
else
{
	// если условие неверно
}
	\end{code}
\end{frame}

\begin{frame}
	\frametitle{Операторы сравнения}
	\begin{itemize}
		\item \lcode{==} \ равно (не путать с =) 
		\item \lcode{!=} \ не равно
		\item \lcode{<} \ \ \ меньше
		\item \lcode{>} \ \ \ больше
		\item \lcode{<=} \ меньше либо равно
		\item \lcode{>=} \ больше либо равно
		\item \lcode{&&} \ логическое И
		\item \lcode{||} \ логическое ИЛИ
	\end{itemize}
\end{frame}

\begin{frame}
	\frametitle{Задачи}
	\prblm{Поменять местами значения двух переменных.}
	\prblm{Решить предыдущую задачу без дополнительной переменной.}
	\prblm{Вычислить модуль введённого числа.}
	\prblm{Определить, является ли введённое число чётным.}
	\prblm{Найти максимальное из двух чисел.}
	\prblm{Проверить, могут ли 3 заданных числа быть сторонами треугольника.}
\end{frame}

\begin{frame}
	\frametitle{Калькулятор}
	Напишите калькулятор, выполняющий одно из 4 арифметических действий над двумя заданными вещественными числами.
	\inp
	На вход подаётся вещественное число a, символ s и вещественное число b ($|a|,|b| \leq 1000, c \in \{\text{'+', '-', '*', '/'}\}$).
	\out
	Напечатайте одно число --- результат вычисления, либо сообщение об ошибке.
	\begin{ex}
	2+3 & 5 \tb
	3.14*2.72 & 8.5408 \tb
	42/0 & Ошибка \tb
	\end{ex}
\end{frame}

\begin{frame}
	\frametitle{Логические выражения}
	\begin{itemize}
		\item \lcode{bool} --- логический тип данных
		\item \lcode{!} --- оператор отрицания
		\item \lcode{0 == false, !0 == true // любое число, кроме нуля}
		\item \lcode{a = !!a; // 0, если a была 0; иначе 1}
		\item Операторы сравнения возвращают результат логического типа
	\end{itemize}
\end{frame}

\begin{frame}
	\frametitle{МКАД}
	Длина Московской кольцевой автомобильной дороги  --- 109 километров. Байкер Вася
	стартует с первого километра МКАД и едет со скоростью \lcode{v} километров в час. На
	какой отметке он остановится через \lcode{t} часов?
	\inp
	На вход подаются два целых числа \lcode{t} и \lcode{v} ($0 \leq t,v \leq 10000$).
	\out
	Напечатайте единственное число от 1 до 109 --- километр МКАД, на котором остановится Вася.
	\begin{ex}
		60 2 & 12 \tb
		109 42 & 1 \tb
		0 146 & 1 \tb
	\end{ex}
\end{frame}

\end{document}
